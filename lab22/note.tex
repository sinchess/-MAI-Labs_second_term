\documentclass[a5paper,12pt]{book}
\usepackage[left=1.7cm,right=0.7cm,top=2cm,bottom=1cm]{geometry}
\usepackage[utf8]{inputenc}
\usepackage[english,russian]{babel}
\usepackage{amssymb}
\usepackage{amsmath,mathtools}
\usepackage{fancyhdr}
\usepackage{scrextend}
\usepackage{wasysym}
\pagestyle{fancy}
\fancyhf{}
\fancyhead[LO]{\scriptsize 140}
\fancyhead[CO]{\scriptsize ЧИСЛЕННОЕ ИНТЕГРИРОВАНИЕ}
\fancyhead[RO]{\scriptsize [ГЛ. III}
\fancyhead[LE]{\scriptsize \S\ 8]}
\fancyhead[CE]{\scriptsize ОПТИМАЛЬНЫЕ КВАДРАТУРЫ}
\fancyhead[RE]{\scriptsize 141}
\renewcommand{\headrulewidth}{0pt}
\linespread{0.8}
\headsep 2mm
\begin{document}
	\noindent {\textit{на классе F} называют величину}
	\vspace{-1mm}
	\footnotesize $$R_N(F) = \sup_{f \in F}|\ R_N(f)\ |,$$\normalsize
	\vskip -3mm
	\noindent где, как обычно,
	\vspace{1mm} \small$$R_N(f) = I(f) - S_N(f).$$

	\vspace{-2mm} Нижняя грань
	\vspace{2mm}
	\footnotesize $$W_N(F)=\inf_{D_j\ P_j} R_N(F)$$\normalsize
	\noindent называется \textit{оптимальной оценкой погрешности квадратур на рассматриваемом классе}.
	Если существует квадратура, для которой $R_N(F)=W_N(F)$, то такую квадратуру
	называют \textit{оптимальной}, или \textit{наилучшей}, \textit{на рассматриваемом классе}.

	Обратимся к случаю
	\vspace{-2mm}
	\footnotesize $$I(f)=\varint\limits_{0}^{1}f(x)dx, \ \ \ f \in F=C_1(A,[0,1]);$$ \normalsize
	\vskip -2mm

	\noindent $C_1(A,[0,1])$ --- класс непрерывных функций с кусочно-непрерыв\-ной
	первой производной, удовлетворяющей условию $|f'(x)\ | \leqslant A$.

	Квадратурную сумму записываем в виде
	\vspace{-2mm}
	\footnotesize $$S_N(f)=\sum\limits_{j=1}^{N} D_jf(x_j),\ \ x_1<...<x_N.$$ \normalsize
	\vskip -2mm
	\noindent Потребуем выполнения условия
	\vspace{-2mm}
	\footnotesize
	\begin{equation}
		\sum_{j=1}^{N}D_j=1.
	\end{equation}
	\normalsize
	\vskip -3mm
	\noindent Если (1) не выполнено, то при $f(x)=c=const$ имеем
	\vspace{0mm}
	\footnotesize $$R_N(f)=\left(1-\sum\limits_{j=1}^{N}D_j\right)c\neq 0.$$ \normalsize
	\vskip 0mm
	\noindent Все функции $f(x)=const$ принадлежат рассматриваемому классу, и следовательно,
	\vspace{-2mm}
	\footnotesize $$R_N(F)\geqslant\sup_c\left(\left|1-\sum\limits_{j=1}^{N}D_j\right|\left|c\right|\right)=\infty.$$ \normalsize
	Ясно, что об оптимальности такой квадратуры говорить не при\-ходится.
	Квадратура, удовлетворяющая условию (1), точна для постоянных, т. е. всех
	многочленов нулевой степени. \ Для таких \\ квадратур, согласно (2.6), справедливо соотношение
	\footnotesize$$R_N(C_1(A,[0,1]))=A\varint\limits_{0}^{1}|\ K_N(y)\ |dy,$$ \normalsize
	\newpage
	\noindent где
	\footnotesize$$K_N(y)=1-y-\sum_{j=1}^{N}D_j(\overline{x_j-y})^0,$$\normalsize
	\footnotesize
	\begin{equation*}
		(\overline{t})^0 = 
		 \begin{cases}
		   1 &\text{при $t > 0,$}\\
		   0 &\text{при $t \leqslant 0.$}
		 \end{cases}
	\end{equation*}
	\normalsize
	\noindent Для $x_{m-1} \leqslant y < x_m$ имеем
	\footnotesize
	\begin{equation*}
		(\overline{x_j-y})^0 =
		 \begin{cases}
		   1 &\text{при $j < m,$}\\
		   0 &\text{при $j \geqslant m,$}
		 \end{cases}
	\end{equation*}
	\normalsize
	\noindent (здесь мы положили $0^0=0$); поэтому
	\footnotesize
	\begin{multline}
		\boldsymbol{\sum_{j=1}^{N}}D_j(\overline{x_j-y})^0=\\
		=
		\begin{cases}
			\text{\ \ \ \ \ \ }1 \text{\ \ \ \ \ \ \ \ \ \ 
			\ \ \ \ \ \ при\ \ \ \ \ \ \ \ } y<x_1,\\
			\ Q_m=\sum\limits_{j=m}^{N}D_j \text{\ \ \ \ при\ \ \ \ }
			x_{m-1}\leqslant y <x_m,\ \ \ m=2,...,N, \\
			\text{\ \ \ \ \ \ \ }0 \text{\ \ \ \ \ \ \ \ 
			\ \ \ \ \ \ \ при\ \ \ \ \ \ \ }x_N \leqslant y.
		\end{cases}
	\end{multline}
	\normalsize

	Мы \ видим, что исходная задача минимизации погрешности
	на классе свелась к решению \ слудующей задачи: \ приблизить
	наилучшим \ \ образом \ в \ метрике $L_1:||\ f\ ||=\varint\limits_{0}^{1}|\ f(x)\ |dx$ функцию \\
	$1-y$ функциями вида (2). Обозначим
	\vspace{-7mm}
	\footnotesize
	\begin{center}
		$$\varint\limits_{0}^{1}|\ K_N(y)\ |dy\ \text{\ \ \ через\ \ \ }\ V(Q_2,...,Q_N; x_1,...,x_N)$$
	\end{center}
	\normalsize
	\vskip -3mm
	Имеем равенство
	\footnotesize
	\begin{multline}
		V(Q_2,...,Q_N; x_1,...,x_N)=\\
		=\varint\limits_{0}^{x_1}|\ y\ |dy +\sum_{m=2}^{N}\ \varint\limits_{x_{m-1}}^{x_m}|\ (1-y)-Q_m\ |dy+\varint\limits_{x_N}^{1}|\ 1-y\ |dy.
	\end{multline}
	\normalsize
	От фиксированного $Q_m$ зависит только одно слагаемое
	\footnotesize
	$$V(Q_m)=\varint\limits_{x_{m-1}}^{x_m}|\ (1-y)-Q_m\ |dy.$$
	\normalsize
\end{document}
